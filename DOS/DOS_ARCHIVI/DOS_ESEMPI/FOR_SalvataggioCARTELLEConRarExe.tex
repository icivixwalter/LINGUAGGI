\documentclass[10pt,a4paper]{book}
\usepackage[utf8]{inputenc}
\usepackage{amsmath}
\usepackage{amsfonts}
\usepackage{amssymb}
\usepackage{graphicx}
\begin{document}
	\subsection{FOR Ciclo di SALVAGGIO CARTELLE CON RAREXE}
	\label{FOR:SALVATAGGI-DirectoryConRAR}{....CICLO FOR DI SALVATAGGI CON RAR.EXE}
	Esempio comprimi le directory e sottodirectory con un CICLO FOR che costruisce
	una directory XAA\_SALVATAGGI nel quale comprimere con il comando rar tutte le
	directory. E' importante applicare il ciclo solo sulle directory create in quella
	corrente. Non salva i file esterni e se non ci sono directory già create non salva nulla.
	\\
	ESEMPIO:\\
	
	\detokenize{
		
		echo off
		
		:----------------------------SET DELLE VARIABILI PATH DI PARTENZA ED ARRIVO
		
		@REM Se non esiste la cartella XX_SALVATAGGI, la crea.
		if not exist "%cd%\XX_SALVATAGGI\" (
		echo Crea la cartella XX_SALVATAGGI
		mkdir %cd%\XX_SALVATAGGI
		)
		
		set PATH_ARRIVO_s="%cd%\XX_SALVATAGGI\"
		
		echo. 
		echo controllo della path di ARRIVO:
		echo "PATH ARRIVO: " %PATH_ARRIVO_s%
		
		
		:----------------------CICLO_FOR= Ciclo For su ogni riga del comando DIR (senza dettagli con le sottocartelle)
		echo Prova DIR
		
		@REM L'unico modo per salvare SOLO le cartelle e non i file. Se vuoi salvare anche i file RIMUOVI /A:d
		for /f "tokens=*" %%G in ('dir /B /A:d %cd%') do (
		
		@REM Per ogni cartella diversa da XX_SALVATAGGI...
		echo controllo directory in esame %%G
		echo controllo path da escludere %PATH_ARRIVO_s%
		IF NOT %%~nG==XX_SALVATAGGI (
		echo --------------  Estraggo "%%~nG" e salvo nella path di arrivo.
		@REM non ci deve essere spazio altrimenti il nome del file viene con lo spazio SI=%PATH_ARRIVO_s%%%~nG ... NO=%PATH_ARRIVO_s% %%~nG
		C:\CASA\Rar.exe a -ep1 "%PATH_ARRIVO_s%%%~nG" "%%~fG"
		)
		)
		eof
		
		
		@REM //NOTE DI FUNZIONAMENTO
		@REM //============================================================================//
		@REM attenzione per il salvataggio sono stati utilizzati questi parametri:
		@REM dir 	/b = elenca il contenuto della cartella corrente senza dettagli
		@REM 	/s = Mostra il percorso completo di ogni file/cartella E DELLE SOTTOCARTELLE!!!!
		@REM 	/a:d = include SOLO LE CARTELLE (escluso perche altrimenti salvava ogni singolo
		@REM 	file della sottocartella in un zip)
		@REM COMANDO rar
		@REM rar.exe	-ep1= esclude la path nel .zip e inserisce solo il nome della sottocartella nell'archivio.	
		
		
		
		
		
	}
	
\end{document}